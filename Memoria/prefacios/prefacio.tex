\chapter*{}
%\thispagestyle{empty}
%\cleardoublepage

%\thispagestyle{empty}

%\input{portada/portada_2}



\cleardoublepage
\thispagestyle{empty}

\begin{center}
{\large\bfseries Sistema de realidad virtual para manipulación remota de brazos robóticos}\\
\end{center}
\begin{center}
David Heredia Cortés\\
\end{center}

%\vspace{0.7cm}
\noindent{\textbf{Palabras clave}: \myKeywords}\\

\vspace{0.7cm}
\noindent{\textbf{Resumen}}\\

El presente trabajo de fin de grado tiene como principales  objetivos la investigación y testeo del control remoto de robots colaborativos en un entorno dinámico e inaccesible para seres humanos. En este contexto, nos centraremos en el uso de dispositivos con altas capacidades hápticas para su integración en las tareas de teleoperación, haciendo la experiencia del ejecutante más sencilla e inmersiva.

En este documento se describe, además del entorno en que se ha pensado aplicar como solución esta tecnología, los costos asociados, la metodología con la que se ha desarrollado el proyecto y los dispositivos de los que se ha hecho uso. A través de los capítulos en los que se divide la memoria, se describirán también las tecnologías empleadas para cada propósito específico y la justificación de la elección de cada una  de las mismas, detallando su papel dentro del trabajo de investigación realizado.

Asimismo, se comentará de forma más detallada cada una de las interfaces con las que hemos lidiado para conseguir coordinar la actividad de los dispositivos. Aclararemos igualmente los problemas que se han presentado durante la realización del proyecto y las soluciones adoptadas, acompañadas del razonamiento que se ha seguido para su implementación. 

Por otra parte, haremos un análisis de los requisitos que se pretendían cubrir, para más adelante exponer el grado de consecución de los objetivos propuestos junto a los test realizados y los resultados obtenidos.

Finalmente, expondremos una disertación del trabajo realizado discutiendo la solución obtenida, junto con una revisión de los cambios en los objetivos y prioridades del estudio, concluyendo con ideas y comentarios para el trabajo futuro dentro de esta línea de investigación.


\cleardoublepage


\thispagestyle{empty}


\begin{center}
{\large\bfseries \myTitleEnglish}\\
\end{center}
\begin{center}
David Heredia Cortés
\end{center}

%\vspace{0.7cm}
\noindent{\textbf{Keywords}: \myKeywordsEnglish}\\

\vspace{0.7cm}
\noindent{\textbf{Abstract}}\\

The main purpose of this final degree project is the investigation and testing of the remote control of collaborative robots in a dynamic and inaccessible environment for human beings. In this context, we will focus on the use of devices with high haptic capabilities for their integration into teleoperation tasks, making the performer's experience simpler and more immersive.

This document describes, as well as the environment in which it has been thought to apply this technology as a solution, as the associated costs, the way in which the project has been developed and the devices that have been used. Through the chapters in which the report is divided, the technologies used for each specific purpose and the justification for the choice of each of them will also be described, explaining their role within the work carried out.

Likewise, the methodology which the project has been carried out and, in detail, each of the interfaces that we have dealt with to coordinate the activity of the devices will be discussed. We will also clarify the problems that have arisen during the implementation of the project and the solutions adopted, adding the reasoning that has been followed for their implementation.

On the other hand, we will make an analysis of the requirements that were intended to be covered, to later expose the degree of achievement of the proposed objectives with the tests carried out and the results obtained.

Finally, we will present a reflection of the carried out work, discussing the solution obtained with a review of the changes in the objectives and priorities of the study, concluding with ideas and comments for future work within this line of research.

\chapter*{}
\thispagestyle{empty}

\noindent\rule[-1ex]{\textwidth}{2pt}\\[4.5ex]

Yo, \textbf{David Heredia Cortés}, alumno de la titulación de Ingeniería Informática de la \textbf{Escuela Técnica Superior
de Ingenierías Informática y de Telecomunicación de la Universidad de Granada}, con DNI 76738338B, autorizo la
ubicación de la siguiente copia de mi Trabajo Fin de Grado en la biblioteca del centro para que pueda ser
consultada por las personas que lo deseen.

\vspace{1cm}

\begin{figure}[hbt]
	\centering
	\includegraphics[width=0.35\textwidth]{imagenes/firmaDavidHeredia.jpg}
\end{figure}

\noindent Fdo: David Heredia Cortés

\vspace{2cm}

\begin{flushright}
Granada a \today.
\end{flushright}


\chapter*{}
\thispagestyle{empty}

\noindent\rule[-1ex]{\textwidth}{2pt}\\[4.5ex]

D. \textbf{\myProf}, Profesor del Área de Arquitectura y Tecnología de Computadores del Departamento de Arquitectura y Tecnología de Computadores de la Universidad de Granada.


\vspace{0.5cm}

\textbf{Informan:}

\vspace{0.5cm}

Que el presente trabajo, titulado \textit{\textbf{\myTitle}},
ha sido realizado bajo su supervisión por \textbf{\myName}, y autorizamos la defensa de dicho trabajo ante el tribunal
que corresponda.

\vspace{0.5cm}

Y para que conste, expiden y firman el presente informe en Granada a \today.

\vspace{1cm}

\textbf{Los directores:}

\vspace{5cm}

\noindent \textbf{\myProf}

\chapter*{Agradecimientos}
\thispagestyle{empty}

       \vspace{1cm}


\textit{A mis padres, por alentar mi progreso en la senda del saber y apoyar mi formación, pese a no haberla podido tener.}

\textit{A mis amigos, por estar junto a mi en los momentos de felicidad y tristeza en cada uno de los que han sido mis mejores años vividos.}

\textit{A mi familia, los que perviven aquí y en mi recuerdo, por celebrar mis éxitos como propios y resultar una fuente de sabiduría y apoyo inagotable.}

