\begin{thebibliography}{2}
	\bibitem {1} G.R. Schmidt, G.A. Landis, and S.R. Oleson "HERRO Missions to Mars and Venus Using Telerobotic Exploration from Orbit" [Accessed 2 September 2021]. see also: S.R. Oleson, G.A. Landis, M. McGuire and G.R. Schmidt HERRO Missions to Mars Using Telerobotic Surface Exploration from Orbit, Journal of the British Interplanetary Society (2012), and HERRO  [Accessed 2 September 2021].
	\bibitem {2} Mars.nasa.gov. 2021. Mission Overview. [online] Available at: $<$\url{https://mars.nasa.gov/mars2020/mission/overview/}$>$ [Accessed 2 September 2021].
	\bibitem {3} Zotovic, R., Mellado, M., Jornet, V., Catret, J. V., \& Puig, D. (2001). Diseño, Implementación y Control de un Sistema Háptico con Realimentación Sensorial en Tele-Robótica. 2º Cong. Int. Interacción Persona-Ordenador.
	\bibitem {4} Hemal, A. and Menon, M., 2011. Robotics in genito-urinary surgery. New York: Springer.
	\bibitem {5} Estayno, M. G., Bauer, J., Guardiola, C., \& Serra, D. (2009). La telepresencia, la teleoperación y la generación de competencias en el Marco de Sistemas CIM (Computer Integrated Manufacturing). In XI Workshop de Investigadores en Ciencias de la Computación.
	\bibitem {6} Montes Franceschi, H. (2019). Análisis cinemático del brazo robótico de Stanford.
	\bibitem {7} Hokayem, P. F., \& Spong, M. W. (2006). Bilateral teleoperation: An historical survey. Automatica, 42(12), 2035-2057.
	\bibitem {8} Martínez, J. M. B. (2016). Desarrollo de tecnologías de telemanipulación con alto grado de escalado orientadas a la interacción hombre-robot en entornos nucleares (Doctoral dissertation, Universidad Politécnica de Madrid).
	\bibitem {9} FAAC. 2021. Military Training Simulators | Air \& Army Training Simulation | FAAC. [online] Available at: $<$\url{https://www.faac.com/training-simulators/military/}$>$ [Accessed 2 September 2021].
	\bibitem {10} Robonaut.jsc.nasa.gov. 2021. Robonaut2. [online] Available at: $<$\url{https://robonaut.jsc.nasa.gov/R2/}$>$ [Accessed 2 September 2021].
	\bibitem {11} Es.wikipedia.org. 2021. Exploración de Venus - Wikipedia, la enciclopedia libre. [online] Available at: $<$\url{https://es.wikipedia.org/wiki/Exploraci\%C3\%B3n\_de\_Venus}$>$ [Accessed 2 September 2021].
	\bibitem {12} Rabanales Sotos, J., Párraga Martínez, I., López-Torres Hidalgo, J., Andrés Pretel, F., \& Navarro Bravo, B. (2011). Tecnologías de la Información y las Telecomunicaciones: Telemedicina. Revista Clínica de Medicina de Familia, 4(1), 42-48.
	\bibitem {13} 2021. [online] Available at: $<$\href{https://www.sdpnoticias.com/estilo-de-vida/video-desde-londres-medico-opera-platano-estados-unidos.html}{https://www.sdpnoticias.com/estilo-de-vida/video-desde-londres-medico-opera-platano-estados-unidos.html}$>$ [Accessed 2 September 2021].
	\bibitem {14} Konesky, G. A. (2002, January). Large-scale teleoperation approach to exploration of the Hudson submarine canyon. In Remote Sensing of the Ocean and Sea Ice 2001 (Vol. 4544, pp. 193-200). International Society for Optics and Photonics.
	\bibitem {15} Álvarez, B., Iborra, A., Alonso, A., y de la Puente, JA (2001). Arquitectura de referencia para la teleoperación de robots: detalles de desarrollo y uso práctico. Práctica de ingeniería de control , 9 (4), 395-402.
	\bibitem {16} Greengard, S. (2019). Realidad virtual . Mit Press.
	\bibitem {17} Furht, B. (Ed.). (2011). Manual de realidad aumentada . Springer Science \& Business Media.
	\bibitem {18} Speicher, M., Hall, BD y Neveling, M. (mayo de 2019). ¿Qué es la realidad mixta ?. En Actas de la Conferencia CHI de 2019 sobre factores humanos en sistemas informáticos (págs. 1-15).
	\bibitem {19} Kristoffersson, A., Coradeschi, S. y Loutfi, A. (2013). Una revisión de la telepresencia robótica móvil. Avances en la interacción persona-computadora , 2013 .
	\bibitem {20} Jones, S. y Dawkins, S. (2018).El sensorama revisitado: evaluación de la aplicación de información multisensorial sobre el sentido de presencia en una película inmersiva de 360 grados en la realidad virtual. En Realidad aumentada y realidad virtual (págs. 183-197). Springer, Cham.
	\bibitem {21} En.wikipedia.org. 2021. The Sword of Damocles (virtual reality) - Wikipedia. [online] Available at: $<$\url{https://en.wikipedia.org/wiki/The\_Sword\_of\_Damocles\_(virtual\_reality)}$>$ [Accessed 2 September 2021].
	\bibitem {22} Reiss, S. M. (1992). VIRTUAL IS SEEING BELIEVING REALITY. Optics and Photonics News, 3(4), 16-21.
	\bibitem {23} Zimmerman, TG, Lanier, J., Blanchard, C., Bryson, S. y Harvill, Y. (1986). Un dispositivo de interfaz de gestos de mano. Boletín ACM Sigchi , 18 (4), 189-192.
	\bibitem {24} Breñosa Martínez, J., 2021. Desarrollo de Tecnologías de Telemanipulación con Alto Grado de Escalado Orientadas a la Interacción Hombre-Robot en Entornos Nucleares. [ebook] Available at: $<$\url{http://oa.upm.es/39597/1/JOSE\_MANUEL\_BRENOSA\_MARTINEZ.pdf$}$>$ [Accessed 2 September 2021].
	\bibitem {25} FAAC. 2021. Air Combat Simulation Training Solutions | FAAC. [online] Available at: $<$\url{https://www.faac.com/simulation-training/military/air-combat-training/}$>$ [Accessed 2 September 2021].
	\bibitem {26} FAAC. 2021. Military Training Simulators | Air \& Army Training Simulation | FAAC. [online] Available at: $<$\url{https://www.faac.com/training-simulators/military/}$>$ [Accessed 2 September 2021].
	\bibitem {27} NASA. 2021. NASA Tests Mixed Reality, Scientific Know-How, and Mission Operations. [online] Available at: $<$\url{https://www.nasa.gov/feature/ames/analog-missions-mixed-reality}$>$ [Accessed 2 September 2021].
	\bibitem {28} Hospimedica.es. 2021. Plataforma de entrenamiento de RV disminuye los errores quirúrgicos críticos a la mitad. [online] Available at: $<$\href{https://www.hospimedica.es/tecnicas-quirurgicas/articles/294786664/plataforma-de-entrenamiento-de-rv-disminuye-los-errores-quirurgicos-criticos-a-la-mitad.html}{https://www.hospimedica.es/tecnicas-quirurgicas/articles/294786664/plataforma-de-entrenamiento-de-rv-disminuye-los-errores-quirurgicos-criticos-a-la-mitad.html}$>$ [Accessed 2 September 2021].
	\bibitem {29} Ahmadpour, N., Randall, H., Choksi, H., Gao, A., Vaughan, C. and Poronnik, P., 2019. Virtual Reality interventions for acute and chronic pain management. The International Journal of Biochemistry \& Cell Biology, 114, p.105568.
	\bibitem {30} Gupta, A., Scott, K. and Dukewich, M., 2017. Innovative Technology Using Virtual Reality in the Treatment of Pain: Does It Reduce Pain via Distraction, or Is There More to It?. Pain Medicine, 19(1), pp.151-159.
	\bibitem {31} Pourmand, A., Davis, S., Marchak, A., Whiteside, T. and Sikka, N., 2018. Virtual Reality as a Clinical Tool for Pain Management. Current Pain and Headache Reports, 22(8).
	\bibitem {32} Dunn, J., Yeo, E., Moghaddampour, P., Chau, B. and Humbert, S., 2017. Virtual and augmented reality in the treatment of phantom limb pain: A literature review. NeuroRehabilitation, 40(4), pp.595-601.
	\bibitem {33} Andreu Toribio, V. and Torronteras López, A., 2015. Introducción a la Háptica. Nuevos dispositivos de entrada y salida. [ebook] Available at: $<$\url{https://core.ac.uk/download/pdf/41822655.pdf}$>$ [Accessed 2 September 2021].
	\bibitem {34} Díaz Tribaldos, Mónica Rocío, Escobar ocampo, José Manuel y Vivas Albán, Óscar Andrés. (2015). INTERFAZ HÁPTICA TIPO GARANTÍA CON REALIMENTACIÓN VIBRATORIA. Revista EIA , (23), 29-39. Obtenido el 2 de septiembre de 2021 de \url{http://www.scielo.org.co/scielo.php?script=sci\_arttext\&pid=S1794-12372015000100003\&lng=en\&tlng=es}.
	\bibitem {35} Sánchez, G., 2009. El Sol: un reactor termonuclear a 150 millones de kilómetros. [ebook] Available at: $<$\url{https://diarium.usal.es/guillermo/files/2014/02/ELSolReactorTermonuclerar24Nov2009.pdf}$>$ [Accessed 2 September 2021].
	\bibitem {36} Shultis, J. and Faw, R., 2005. Fundamentals of nuclear science and engineering. [Boca Raton, Fla.]: CRC Press.
	\bibitem {37} Blaum, K. (2006). Espectrometría de masas de alta precisión con iones almacenados. Physics Reports , 425 (1), 1-78.
	\bibitem {38} Tsokos, K., 2010. Physics for the IB Diploma. Cambridge: Cambridge University Press.
	\bibitem {39} Günther, H. and Müller, V., n.d. The special theory of relativity.
	\bibitem {40} En.wikipedia.org. 2021. Coulomb barrier - Wikipedia. [online] Available at: $<$\url{https://en.wikipedia.org/wiki/Coulomb\_barrier}$>$ [Accessed 2 September 2021].
	\bibitem {41} En.wikipedia.org. 2021. Strong interaction - Wikipedia. [online] Available at: $<$\url{https://en.wikipedia.org/wiki/Strong\_interaction}$>$ [Accessed 2 September 2021].
	\bibitem {42} Claus E. Rolfs; William S. Rodney (1988). Cauldrons in the Cosmos. The University of Chicago Press. p. 354.
	\bibitem {43} Kotschenreuther, M., Valanju, P., Mahajan, S. and Schneider, E., 2009. Fusion–Fission Transmutation Scheme—Efficient destruction of nuclear waste. Fusion Engineering and Design, 84(1), pp.83-88.
	\bibitem {44} Csn.es. 2021. Fusión nuclear - CSN. [online] Available at: $<$\url{https://www.csn.es/fusion-nuclear}$>$ [Accessed 2 September 2021].
	\bibitem {45} Energy.gov. 2021. DOE Explains...Deuterium-Tritium Fusion Reactor Fuel. [online] Available at: $<$\url{https://www.energy.gov/science/doe-explainsdeuterium-tritium-fusion-reactor-fuel}$>$ [Accessed 2 September 2021].
	\bibitem {46} Ibarra, A., Arbeiter, F., Bernardi, D., Cappelli, M., García, A., Heidinger, R., ... y Tian, K. (2018). El proyecto IFMIF-DONES: anteproyecto de ingeniería. Fusión nuclear , 58 (10), 105002.
	\bibitem {47} Aymar, R., Barabaschi, P. y Shimomura, Y. (2002). El diseño ITER. Física del plasma y fusión controlada , 44 (5), 519.
	\bibitem {48} Zani, L., Bayer, CM, Biancolini, ME, Bonifetto, R., Bruzzone, P., Brutti, C., ... y Zanino, R. (2016). Resumen de los avances en el diseño del sistema magnético del reactor DEMO de la UE. Transacciones IEEE sobre superconductividad aplicada , 26 (4), 1-5.
	\bibitem {49} Valencia-García, R., Lagos-Ortiz, K., Alcaraz-Mármol, G., Del Cioppo, J. and Vera-Lucio, N., n.d. Technologies and innovation.
	\bibitem {50} En.wikipedia.org. 2021. Unreal Engine - Wikipedia. [online] Available at: $<$\url{https://en.wikipedia.org/wiki/Unreal\_Engine}$>$ [Accessed 2 September 2021].
	\bibitem {51} Pharmaceutical-technology.com. 2020. Unreal Engine: from gaming to ground-breaking cures. [online] Available at: $<$\url{https://www.pharmaceutical-technology.com/features/unreal-engine-gaming-ground-breaking-cures/}$>$ [Accessed 2 September 2021].
	\bibitem {52} Vinciguerra, D. and Howell, A., n.d. The GameMaker standard.
	\bibitem {53} Castañeda, C. C. C. (2016). Ros-gazebo. una valiosa Herramienta de Vanguardia para el desarrollo de la robótica. Publicaciones e Investigación, 10, 145-160.
	\bibitem {54} Haas, J. (2014). Una historia del motor de juegos Unitarios. Diss. INSTITUTO POLITÉCNICO WORCESTER .
	\bibitem {55} Craighead, J., Burke, J. y Murphy, R. (2008, septiembre). Usando el motor del juego de la unidad para desarrollar sarge: un estudio de caso. En Actas del Taller de Simulación de 2008 en la Conferencia Internacional sobre Robots y Sistemas Inteligentes (IROS 2008) .
	\bibitem {56} Blischak, JD, Davenport, ER y Wilson, G. (2016). Una introducción rápida al control de versiones con Git y GitHub. Biología computacional PLoS , 12 (1), e1004668.
	\bibitem {57} Smithsonian Magazine. 2021. Here's What the Future of Haptic Technology Looks (Or Rather, Feels) Like. [online] Available at: $<$\href{https://www.smithsonianmag.com/innovation/heres-what-future-haptic-technology-looks-or-rather-feels-180971097/}{https://www.smithsonianmag.com/innovation/heres-what-future-haptic-technology-looks-or-rather-feels-180971097/.}$>$ [Accessed 2 September 2021].
	\bibitem {58} 3D Systems. 2021. Touch X | 3D Systems. [online] Available at: $<$\url{https://www.3dsystems.com/haptics-devices/touch-x}$>$ [Accessed 2 September 2021].
	\bibitem {59} Jiang, Y., Yang, C., Wang, X. y Su, CY (junio de 2016). Modelado cinemático de dispositivo geomagic touch x háptico basado en la identificación de parámetros adaptativos. En 2016 IEEE International Conference on Real-time Computing and Robotics (RCAR) (págs. 295-300). IEEE.
	\bibitem {60} Peshkin, M. y Colgate, JE (1999). Cobots. Robot industrial: una revista internacional .
	\bibitem {61} Senar, J. C. (1999). La medición de la repetibilidad y el error de medida. Etologuía, 17, 53-64.
	\bibitem {62} Ju, Z., Yang, C., \& Ma, H. (2014, July). Kinematics modeling and experimental verification of baxter robot. In Proceedings of the 33rd Chinese control conference (pp. 8518-8523). IEEE.
	\bibitem {63} Fitzgerald, C. (2013, April). Developing baxter. In 2013 IEEE Conference on Technologies for Practical Robot Applications (TePRA) (pp. 1-6). IEEE.
	\bibitem {64} GitHub. 2015. GitHub - RethinkRobotics/baxter: Baxter Research Robot SDK. [online] Available at: $<$\url{https://github.com/RethinkRobotics/baxter}$>$ [Accessed 2 September 2021].
	\bibitem {65} Theobald, M., Sozio, M., Suchanek, F., \& Nakashole, N. (2010). URDF: Efficient reasoning in uncertain RDF knowledge bases with soft and hard rules.
	\bibitem {66} Merino López, R., 2014. CREACIÓN DE MODELO URDF DEL ROBOT MANFRED. [ebook] JAVIER V. GÓMEZ GONZÁLEZ. Available at: $<$\url{https://jvgomez.github.io/files/works/theses/rmerino\_report.pdf}$>$ [Accessed 2 September 2021].
	\bibitem {67} Konrad, A. (2019). Simulation of Mobile Robots with Unity and ROS: A Case-Study and a Comparison with Gazebo.
	\bibitem {68} GitHub. 2021. GitHub - Unity-Technologies/URDF-Importer: URDF importer. [online] Available at: $<$\url{https://github.com/Unity-Technologies/URDF-Importer}$>$ [Accessed 2 September 2021].
	\bibitem {69} D'Souza, A., Vijayakumar, S., \& Schaal, S. (2001, October). Learning inverse kinematics. In Proceedings 2001 IEEE/RSJ International Conference on Intelligent Robots and Systems. Expanding the Societal Role of Robotics in the the Next Millennium (Cat. No. 01CH37180) (Vol. 1, pp. 298-303). IEEE.
	\bibitem {70} Chitta, S., Sucan, I., \& Cousins, S. (2012). Moveit![ros topics]. IEEE Robotics \& Automation Magazine, 19(1), 18-19.
	\bibitem {71} Koubâa, A. (Ed.). (2017). Robot Operating System (ROS) (Vol. 1, pp. 112-156). Cham: Springer.
	\bibitem {72} GitHub. 2014. API Reference · RethinkRobotics/sdk-docs Wiki. [online] Available at: $<$\url{https://github.com/RethinkRobotics/sdk-docs/wiki/API-Reference}$>$ [Accessed 2 September 2021].
	\bibitem {73} 2016. [online] Available at: $<$\url{https://assetstore.unity.com/packages/essentials/tutorial-projects/unity-5-haptic-plugin-for-geomagic-openhaptics-3-3-hlapi-hdapi-34393}$>$ [Accessed 2 September 2021].
	\bibitem {74} 2019. [online] Available at: $<$\url{https://assetstore.unity.com/packages/tools/integration/3d-systems-openhaptics-unity-plugin-134024}$>$ [Accessed 2 September 2021].
	\bibitem {75} Jakobsen, R., \& Larsson, H. (2005). DLL hooks in the Microsoft Windows Operating System.
	\bibitem {76} Dalrymple, P., \& Griffiths, R. (2005). Force, mass and acceleration. Anaesthesia \& Intensive Care Medicine, 6(9), 294.
	\bibitem {77} 2020. OpenHaptics® Toolkit Version 3.5.0 API Reference Guide. [ebook] 3D Systems. Available at: $<$\url{https://s3.amazonaws.com/dl.3dsystems.com/binaries/Sensable/OH/3.5/OpenHaptics\_Toolkit\_API\_Reference\_Guide.pdf}$>$ [Accessed 2 September 2021].
	\bibitem {78} Tipler, P. and Mosca, G., 2004. Physics for scientists and engineers. New York, NY: W.H. Freeman.
	\bibitem {79} Tanne, K., Koenig, H. A., \& Burstone, C. J. (1988). Moment to force ratios and the center of rotation. American Journal of Orthodontics and Dentofacial Orthopedics, 94(5), 426-431.
	\bibitem {80} Guiñón, J. L., Ortega, E., Aviñó, J. G. A., \& Herranz, V. P. (2007). Filtrado de senales (i): Implementación y análisis del filtro de media móvil. Ingeniería química, (448), 220-227.
	\bibitem {81} Nasa.gov. 2021. [online] Available at: $<$\url{https://www.nasa.gov/sites/default/files/styles/side\_image/public/thumbnails/image/iss055e076762.jpg?itok=UUPXxcWH}$>$ [Accessed 2 September 2021].
	\bibitem {82} Our World in Data. 2021. Global primary energy consumption by source. [online] Available at: $<$\url{https://ourworldindata.org/grapher/global-energy-substitution?country=~OWID\_WRL}$>$ [Accessed 2 September 2021].
	\bibitem {83} File:Perseverance selfie, looking at WATSON (PIA24542).jpg. (2021, June 27). Wikimedia Commons, the free media repository. Retrieved 23:04, September 2, 2021 from \url{https://commons.wikimedia.org/w/index.php?title=File:Perseverance\_selfie,\_looking\_at\_WATSON\_(PIA24542).jpg\&oldid=571640818}.
	\bibitem {84} File:Deuterium-tritium fusion.svg. (2021, July 21). Wikimedia Commons, the free media repository. Retrieved 23:13, September 2, 2021 from \url{https://commons.wikimedia.org/w/index.php?title=File:Deuterium-tritium\_fusion.svg\&oldid=57624131}.
	\bibitem {85} Es.wikipedia.org. 2021. Unity (motor de videojuego) - Wikipedia, la enciclopedia libre. [online] Available at: $<$\url{https://es.wikipedia.org/wiki/Unity\_(motor\_de\_videojuego)}$>$ [Accessed 2 September 2021].
	\bibitem {86} Giri, G., Maddahi, Y. and Zareinia, K., 2021. An Application-Based Review of Haptics Technology. Robotics, 10(1), p.29.
	\bibitem {87} Robots colaborativos. Qué es un robot colaborativo, c., 2021.  ¿Qué es un robot colaborativo o cobot? Marcas y precio. [online] REVISTA DE ROBOTS. Available at: $<$\href{https://revistaderobots.com/cobots/cobots-o-robots-colaborativos-caracteristicas-ventajas-y-fabricantes-de-brazos-roboticos-industriales/}{https://revistaderobots.com/cobots/cobots-o-robots-colaborativos-caracteristicas-ventajas-y-fabricantes-de-brazos-roboticos-industriales/}$>$ [Accessed 2 September 2021].
	\bibitem {88} Web.archive.org. 2014. Baxter | Redefining Robotics and Manufacturing | Rethink Robotics. [online] Available at: $<$\url{https://web.archive.org/web/20140826071530/http://www.rethinkrobotics.com/products/baxter/}$>$ [Accessed 2 September 2021].
	\bibitem {89} GitHub. 2021. Unity-Robotics-Hub/unity\_ros.png at main · Unity-Technologies/Unity-Robotics-Hub. [online] Available at: $<$\url{https://github.com/Unity-Technologies/Unity-Robotics-Hub/blob/main/tutorials/ros\_unity\_integration/images/unity\_ros.png}$>$ [Accessed 2 September 2021].
	\bibitem {90} NASA. 2021. Remote Manipulator System (Canadarm2). [online] Available at: $<$\url{https://www.nasa.gov/mission\_pages/station/structure/elements/remote-manipulator-system-canadarm2/}$>$ [Accessed 2 September 2021].
	\bibitem {91} Martín-Barrio, Andrés, et al. "Application of immersive technologies and natural language to hyper-redundant robot teleoperation." Virtual Reality 24.3 (2020): 541-555.
	\bibitem {92} Calvente, A. M. (2007). De la Globalización a la Planetarización. El devenir de la civilización Humana en la búsqueda del enlace sostenible.
	\bibitem{93} Cortés, D. H. (n.d.). Xdavid1999/tfg at main. GitHub. \url{https://github.com/XDavid1999/TFG}. 
\end{thebibliography}